\documentclass{article}
\usepackage[T1]{fontenc}
\usepackage{graphicx}
\usepackage{listings}
\usepackage{hyperref}
\usepackage{xcolor}
\usepackage[utf8]{inputenc}

\lstset{
    basicstyle=\ttfamily\small,
    frame=single,
    backgroundcolor=\color{gray!10},
    keywordstyle=\color{blue},
    commentstyle=\color{green!50!black},
    stringstyle=\color{red},
    numbers=left,
    numberstyle=\tiny\color{gray},
    breaklines=true,
    captionpos=b
}

\title{Zbiór Mandelbrota - Projekt w C i Gnuplot}
\author{Antoni Rucki}
\date{\today}

\begin{document}
\maketitle
\tableofcontents
\newpage
\section{Kod w C}
Poniżej znajduje się kod generujący zbiór Mandelbrota.

\begin{lstlisting}[language=C, caption=Kod programu Mandelbrota]
#include <stdio.h>
#include <complex.h>

#define WIDTH 800
#define HEIGHT 800
#define MAX_ITER 1000

int mandelbrot(double complex c) {
    double complex z = 0;
    int iter = 0;
    while (cabs(z) <= 2.0 && iter < MAX_ITER) {
        z = z * z + c;
        iter++;
    }
    return iter;
}

int main() {
    FILE *fp = fopen("mandelbrot.dat", "w");
    if (!fp) {
        printf("Błąd otwierania pliku!\n");
        return 1;
    }

    for (int y = 0; y < HEIGHT; y++) {
        for (int x = 0; x < WIDTH; x++) {
            double real = (x - WIDTH / 2.0) * 4.0 / WIDTH;
            double imag = (y - HEIGHT / 2.0) * 4.0 / HEIGHT;
            double complex c = real + imag * I;
            int value = mandelbrot(c);
            fprintf(fp, "%d %d %d\n", x, y, value);
        }
    }

    fclose(fp);
    return 0;
}
\end{lstlisting}

\newpage
\section{Kod Gnuplota}
Poniżej znajduje się kod do gnuplota o nazwie mandelbrot.gp.
\begin{lstlisting}[language=Gnuplot, caption=Kod Gnuplota]
set terminal pdfcairo size 800,800
set output 'mandelbrot.pdf'
set view map
unset key
set size ratio -1
set palette defined (0 "black", 1 "blue", 2 "cyan", 3 "green", 4 "yellow", 5 "red", 6 "white")
plot 'mandelbrot.dat' using 1:2:3 with points palette
\end{lstlisting}

\newpage
\section{Wizualizacja zbioru Mandelbrota}
Na poniższym rysunku przedstawiono zbiór Mandelbrota wygenerowany na podstawie kodu w C i Gnuplota.

\begin{figure}[h]
    \centering
    \includegraphics[width=\textwidth]{mandelbrot.pdf}
    \caption{Zbiór Mandelbrota}
    \label{fig:mandelbrot}
\end{figure}

\end{document}
